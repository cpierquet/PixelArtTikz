% !TeX TXS-program:compile = txs:///arara
% arara: pdflatex: {shell: yes, synctex: no, interaction: batchmode}
% arara: pdflatex: {shell: yes, synctex: no, interaction: batchmode} if found('log', '(undefined references|Please rerun|Rerun to get)')

\documentclass{article}
\usepackage[utf8]{inputenc}
\usepackage[T1]{fontenc}
\usepackage{amsmath,amssymb}
\usepackage{PixelArtTikz}
\usepackage{couleurs-fr}
\usepackage{fontawesome5}
\usepackage{enumitem}
\usepackage{tabularray}
\usepackage{multicol}
\usepackage{fancyvrb}
\usepackage{fancyhdr}
\fancyhf{}
\renewcommand{\headrulewidth}{0pt}
\lfoot{\sffamily\small [PixelArtTikz]}
\cfoot{\sffamily\small - \thepage{} -}
\rfoot{\hyperlink{matoc}{\small\faArrowAltCircleUp[regular]}}

%\usepackage{hvlogos}
\usepackage{hologo}
\usepackage{xspace}
\providecommand\tikzlogo{Ti\textit{k}Z}
\providecommand\TeXLive{\TeX{}Live\xspace}
\providecommand\PSTricks{\textsf{PSTricks}\xspace}
\let\pstricks\PSTricks
\let\TikZ\tikzlogo
\newcommand\TableauDocumentation{%
	\begin{tblr}{width=\linewidth,colspec={X[c]X[c]X[c]X[c]X[c]X[c]},cells={font=\sffamily}}
		{\huge \LaTeX} & & & & &\\
		& {\huge \hologo{pdfLaTeX}} & & & & \\
		& & {\huge \hologo{LuaLaTeX}} & & & \\
		& & & {\huge \TikZ} & & \\
		& & & & {\huge \TeXLive} & \\
		& & & & & {\huge \hologo{MiKTeX}} \\
	\end{tblr}
}

\usepackage{hyperref}
\urlstyle{same}
\hypersetup{pdfborder=0 0 0}
\usepackage[margin=1.5cm]{geometry}
\setlength{\parindent}{0pt}

\def\TPversion{0.1.5}
\def\TPdate{31/10/2024}
\usepackage{minted2}
\usepackage[most]{tcolorbox}
\tcbuselibrary{minted}
\NewTCBListing{PresentationCode}{ O{blue} m }{%
	sharp corners=downhill,enhanced,arc=12pt,skin=bicolor,%
	colback=#1!1!white,colframe=#1!75!black,colbacklower=white,%
	attach boxed title to top right={yshift=-\tcboxedtitleheight},title=Code \LaTeX,%
	boxed title style={%
		colframe=#1!75!black,colback=#1!15!white,%
		,sharp corners=downhill,arc=12pt,%
	},%
	fonttitle=\color{#1!90!black}\itshape\ttfamily\footnotesize,%
	listing engine=minted,minted style=colorful,
	minted language=tex,minted options={tabsize=4,fontsize=\footnotesize,autogobble},
	#2
}
\usepackage[english]{babel}
\newcommand\Cle[1]{{\bfseries\sffamily\textlangle #1\textrangle}}

\begin{document}

\pagestyle{fancy}

\thispagestyle{empty}

\vspace{2cm}

\begin{center}
	\begin{minipage}{0.75\linewidth}
	\begin{tcolorbox}[colframe=yellow,colback=yellow!15]
		\begin{center}
			\begin{tabular}{c}
				{\Huge \texttt{PixelArtTikz [en]}}\\
				\\
				{\LARGE PixelArts, with Ti\textit{k}Z}, \\
				\\
				{\LARGE with solution and colors.} \\
			\end{tabular}
			
			\medskip
			
			{\small \texttt{Version \TPversion{} -- \TPdate}}
		\end{center}
	\end{tcolorbox}
\end{minipage}
\end{center}

\vspace{0.5cm}

\begin{center}
	\begin{tabular}{c}
	\texttt{Cédric Pierquet}\\
	{\ttfamily c pierquet -- at -- outlook . fr}\\
	\texttt{\url{https://github.com/cpierquet/PixelArtTikz}}
\end{tabular}
\end{center}

\vspace{0.25cm}

{$\blacktriangleright$~~Commands to display PixelArts.}

\smallskip

{$\blacktriangleright$~~Environment to complete the PixelArt.}

\smallskip

\vspace{1cm}

\begin{center}
\begin{filecontents*}[overwrite]{test1.csv}
-,-,-,-,-,-,4,4,4,4,-,-,-,-,-,-
-,-,-,-,4,4,1,1,1,1,4,4,-,-,-,-
-,-,-,4,1,1,1,1,1,1,1,1,4,-,-,-
-,-,4,1,1,1,1,1,1,1,1,1,1,4,-,-
-,-,4,1,1,1,1,1,1,1,1,1,1,4,-,-
-,4,1,9,9,1,1,1,1,1,1,9,9,1,4,-
-,4,9,9,9,9,4,4,4,4,9,9,9,9,4,-
-,4,9,4,9,9,4,4,4,4,9,4,9,9,4,-
-,4,1,9,9,9,4,4,4,4,9,9,9,1,4,-
-,-,4,1,1,9,4,4,4,4,9,1,1,4,-,-
-,-,4,1,1,1,4,4,4,4,1,1,1,4,-,-
-,-,-,4,1,1,1,4,4,1,1,1,4,-,-,-
-,-,4,3,1,1,1,1,1,1,1,1,3,4,-,-
-,4,6,3,1,1,1,1,1,1,1,1,3,6,4,-
-,4,6,6,1,1,1,1,1,1,1,1,6,6,4,-
-,4,6,6,1,1,1,1,1,1,1,1,6,6,4,-
-,4,6,4,1,1,1,4,4,1,1,1,4,6,4,-
2,2,4,2,4,4,4,2,2,4,4,4,2,4,2,2
2,2,2,2,2,2,2,2,2,2,2,2,2,2,2,2
2,2,2,2,2,2,2,2,2,2,2,2,2,2,2,2
-,-,-,-,-,4,1,1,1,1,4,-,-,-,-,-
-,-,-,-,-,-,4,1,1,4,-,-,-,-,-,-
-,-,-,-,-,-,-,4,4,-,-,-,-,-,-,-
\end{filecontents*}

\PixlArtTikz[Codes=123469,Style=\ttfamily,Unit=0.35]{test1.csv}
~~
\PixlArtTikz[Codes=123469,Symbols={A,B,C,D,E,F},Symb,Style=\ttfamily,Unit=0.35]{test1.csv}
~~
\PixlArtTikz[Codes=123469,Colors={red,brown,yellow,black,blue,white},Correction,Unit=0.35]{test1.csv}
\end{center}

\vspace{0.5cm}

%\hfill{}\textit{Merci aux membres du groupe \faFacebook{} du \og Coin \LaTeX{} \fg{} pour leur aide et leurs idées !}

%\hfill{}\textit{Merci à Denis Bitouzé et à Patrick Bideault pour leurs retours et idées !}

\vfill

\hrule

\medskip

\TableauDocumentation

\medskip

\hrule

\medskip

\newpage

\phantomsection
\hypertarget{matoc}{}

\tableofcontents

\newpage

\part{Introduction}

\section{The package PixelArtTikz}

\subsection{Introduction}

The idea is to \textit{propose}, within a Ti\textit{k}Z environment, a macro to generate PixelArt.

The data is \textit{read} from a \textsf{csv} file, already existing in the folder of the \textsf{tex} file, or created on-the-fly by \textsf{filecontents}.

\medskip

Some advices about the \textsf{cvs} file :

\begin{itemize}
	\item the \textsf{csv} file must use "," as separator;
	\item empty cells are coded by "\texttt{-}".
\end{itemize}

\begin{PresentationCode}{listing only}
\begin{filecontents*}{filename.csv}
	A,B,C,D
	A,B,D,C
	B,A,C,D
	B,A,D,C
\end{filecontents*}
\end{PresentationCode}

While compiling, the file \textsf{filename.csv} will be created, and the option \Cle{[overwrite]} will propagate the modifications!

\subsection{Loading of the package, and option}

The package \textsf{csvsimple} is necessary in order to read the \textsf{csv} file.

The package is available in two versions, one written in \hologo{LaTeX2e} and the other in \hologo{LaTeX3}. By default, \textsf{PixelArtTikz} loads the \hologo{LaTeX3} version, but an \textit{option} is available to work with the \hologo{LaTeX2e} version.

\smallskip

The option \Cle{[csvii]} forces the usage of the \hologo{LaTeX2e} version.

\begin{PresentationCode}{listing only}
\usepackage{PixelArtTikz}                     %package latex3
%which loads
%\RequirePackage{expl3}
%\RequirePackage[l3]{csvsimple}

\usepackage[csvii]{PixelArtTikz}              %package latex2
%which loads
%\RequirePackage[legacy]{csvsimple}
\end{PresentationCode}

\subsection{Used packages}

It's fully compatible with usual \hologo{LaTeX} engines, such as \textsf{latex}, \textsf{pdflatex}, \textsf{lualatex} or \textsf{xelatex}.

\medskip

It loads the following packages and libraries:

\begin{itemize}
	\item \texttt{tikz}, \texttt{xintexpr} et \texttt{xinttools};
	\item \texttt{xstring}, \texttt{xparse}, \texttt{simplekv} and \texttt{listofitems}.
\end{itemize}

\pagebreak

\subsection{Macros and environment}

There are two ways to create PixelArt:

\begin{itemize}
	\item with an independent macro;
	\item with a Ti\textit{k}Z environment in order to add code afterwards.
\end{itemize}

\begin{PresentationCode}{listing only}
%Independent macro
\PixlArtTikz[keys]<options tikz>{file.csv}

%Semi-independent macro, in a tiks environment
\PixlArtTikz*[keys]{file.csv}

%environment
\begin{EnvPixlArtTikz}[keys]<options tikz>{file.csv}
	%tikz code
\end{EnvPixlArtTikz}
\end{PresentationCode}

\section{Colors}

Concerning colors: the user can use all colors provided by loaded packages!

\smallskip

Without extra packages, the available colors are:

\smallskip

\begin{tblr}{width=\linewidth,colspec={*{10}{X[m,c]}},cells={font=\small\ttfamily}}
	\SetCell{bg=magenta} magenta &
	\SetCell{bg=cyan} cyan &
	\SetCell{bg=blue,fg=white} blue &
	\SetCell{bg=green} green &
	\SetCell{bg=red} red &
	\SetCell{bg=darkgray,fg=white} darkgray &
	\SetCell{bg=olive,fg=white} olive &
	\SetCell{bg=lime} lime &
	\SetCell{bg=brown} brown &
	\SetCell{bg=lightgray} lightgray \\
	\SetCell{bg=white} white &
	\SetCell{bg=gray,fg=white} gray &
	\SetCell{bg=black,fg=white} black &
	\SetCell{bg=yellow} yellow &
	\SetCell{bg=violet,fg=white} violet &
	\SetCell{bg=teal,fg=white} teal &
	\SetCell{bg=purple,fg=white} purple &
	\SetCell{bg=pink} pink &
	\SetCell{bg=orange} orange & \\
\end{tblr}

\pagebreak

\part{Macros and environment}

\section{Main macro}

\subsection{Example}

The macro \texttt{\textbackslash PixlArtTikz} needs :

\begin{itemize}
	\item the file \textsf{csv};
	\item the list (by a string) of codes used in the file \textsf{csv} (e.g.\ \texttt{234679} or \texttt{ABCDJK}\ldots);
	\item the list of symbols (if needed) to print in the cells, e.g.\ \texttt{25,44,12} or \texttt{AA,AB,AC};
	\item the list of colors (for the correction), same order as the codes.
\end{itemize}

We can begin by creating the file \textsf{csv}, directly within the \textsf{tex} code, or with a external file.

\begin{PresentationCode}{}
%creation of the csv
\begin{filecontents*}[overwrite]{base.csv}
	A,B,C,D
	A,B,D,C
	B,A,D,C
	C,A,B,D
\end{filecontents*}
\end{PresentationCode}

\begin{PresentationCode}{}
%instructions and pixelarts
\begin{center}
	\begin{tblr}{colspec={*{4}{Q[1.25cm,c,m]}},hlines,vlines,rows={1.15em}}
		\SetCell[c=4]{c} Instructions & & & \\
		A & B & C & D \\
		45 & 22 & 1 & 7 \\
		Black & Green & Yellow & Red \\
	\end{tblr}
\end{center}

\PixlArtTikz[Codes=ABCD,Style=\large\sffamily,Unit=0.85]{base.csv}
~~
\PixlArtTikz[Codes=ABCD,Symbols={45,22,1,7},Symb,Style=\large\sffamily,Unit=0.85]{base.csv}
~~
\PixlArtTikz[Codes=ABCD,Colors={black,green,yellow,red},Correction,Unit=0.85]{base.csv}
~~
\PixlArtTikz[Codes=ABCD,Colors={black,green,yellow,red},Correction,Border=false,Unit=0.85]{base.csv}
\end{PresentationCode}

\pagebreak

\subsection{Options and keys}

\begin{PresentationCode}{listing only}
\PixlArtTikz[keys]<options tikz>{file.csv}
\end{PresentationCode}

The first argument, \textit{optional} and between \texttt{[...]} proposes the \textsf{keys}:

\begin{itemize}
	\item the key \Cle{Codes} with the \textit{string} of \textit{simple} codes of the \textsf{csv} file;
	\item the key \Cle{Colors} with the \textit{list} of colors;
	\item the key \Cle{Symbols} with the \textit{optional list} of alt.\ symbols for the cells;
	\item the boolean \Cle{Correction} to color the PixelArt;\hfill{}default \textsf{false}
	\item the boolean \Cle{Symb} to print the symbols;\hfill{}default \textsf{false}
	\item the boolean \Cle{Border} to print borders of the cells;\hfill{}default \textsf{true}
	\item the key \Cle{Style} to specifythe style of the text. \hfill{}default \textsf{\textbackslash scriptsize}
\end{itemize}

The second argument, \textit{optional} and between \texttt{<...>}, are Ti\textit{k}Z options to pass on to the environment which creates the PixelArt.

\medskip

The third argument, \textit{mandatory}, is the filename of the \textsf{csv}.


\begin{PresentationCode}{}
%creation of the csv
\begin{filecontents*}[overwrite]{test1.csv}
	-,-,-,-,-,-,4,4,4,4,-,-,-,-,-,-
	-,-,-,-,4,4,1,1,1,1,4,4,-,-,-,-
	-,-,-,4,1,1,1,1,1,1,1,1,4,-,-,-
	-,-,4,1,1,1,1,1,1,1,1,1,1,4,-,-
	-,-,4,1,1,1,1,1,1,1,1,1,1,4,-,-
	-,4,1,9,9,1,1,1,1,1,1,9,9,1,4,-
	-,4,9,9,9,9,4,4,4,4,9,9,9,9,4,-
	-,4,9,4,9,9,4,4,4,4,9,4,9,9,4,-
	-,4,1,9,9,9,4,4,4,4,9,9,9,1,4,-
	-,-,4,1,1,9,4,4,4,4,9,1,1,4,-,-
	-,-,4,1,1,1,4,4,4,4,1,1,1,4,-,-
	-,-,-,4,1,1,1,4,4,1,1,1,4,-,-,-
	-,-,4,3,1,1,1,1,1,1,1,1,3,4,-,-
	-,4,6,3,1,1,1,1,1,1,1,1,3,6,4,-
	-,4,6,6,1,1,1,1,1,1,1,1,6,6,4,-
	-,4,6,6,1,1,1,1,1,1,1,1,6,6,4,-
	-,4,6,4,1,1,1,4,4,1,1,1,4,6,4,-
	2,2,4,2,4,4,4,2,2,4,4,4,2,4,2,2
	2,2,2,2,2,2,2,2,2,2,2,2,2,2,2,2
	2,2,2,2,2,2,2,2,2,2,2,2,2,2,2,2
	-,-,-,-,-,4,1,1,1,1,4,-,-,-,-,-
	-,-,-,-,-,-,4,1,1,4,-,-,-,-,-,-
	-,-,-,-,-,-,-,4,4,-,-,-,-,-,-,-
\end{filecontents*}
\end{PresentationCode}

\begin{PresentationCode}{}
%simple codes
%empty case with -
\PixlArtTikz[Codes=123469,Style=\ttfamily,Unit=0.35]{test1.csv}
~~
\PixlArtTikz[Codes=123469,Colors={red,brown,yellow,black,blue,white},Correction,Unit=0.35]{test1.csv}
~~
\PixlArtTikz[Codes=123469,Colors={red,brown,yellow,black,blue,white},Correction,Unit=0.35,Border=false]{test1.csv}
\end{PresentationCode}

\pagebreak

In the following example, the \textit{symbols} to print can't be used for the \textit{codes}, so we can use the keys \Cle{Symbols} and \Cle{Symb} to bypass this limitation.

\begin{PresentationCode}{}
%symbols associated to codes

\begin{filecontents*}[overwrite]{cap.csv}
	-,-,-,-,-,-,-,-,D,-,D,-,D,-,-,-,-,-,-,-,-,-
	-,D,D,-,-,-,-,D,D,D,D,D,D,-,-,D,D,D,D,-,-,-
	D,-,-,D,-,D,D,F,F,F,F,F,F,D,D,-,-,-,-,D,-,-
	-,D,-,-,D,F,F,F,-,-,F,F,F,F,F,D,-,D,D,-,-,-
	-,-,D,D,F,F,F,-,F,F,-,F,F,F,F,F,D,D,-,-,-,-
	-,-,-,D,F,F,F,F,F,F,F,F,F,F,F,F,D,-,-,-,-,-
	-,-,-,D,F,J,J,J,J,J,J,J,F,F,F,F,D,-,-,-,-,-
	-,-,-,D,J,-,-,-,J,-,-,-,J,J,F,F,D,-,-,-,-,-
	-,-,-,D,J,-,D,-,J,-,D,-,J,J,B,B,D,-,-,-,-,-
	-,-,-,D,J,-,-,-,J,-,-,-,J,J,B,B,D,-,-,-,-,-
	-,-,-,D,C,J,J,J,J,J,J,J,J,C,C,C,D,-,-,-,-,-
	-,-,-,D,C,C,C,C,C,C,C,C,C,C,C,D,D,D,-,-,-,-
	-,-,-,D,C,C,C,D,D,D,D,D,D,C,D,A,A,A,D,-,-,-
	-,-,-,D,F,C,C,C,C,C,C,C,C,D,A,-,-,-,A,D,-,-
	-,-,-,D,F,C,F,C,C,C,C,F,D,A,-,A,A,A,-,A,D,-
	-,-,D,C,F,F,F,F,C,C,F,D,A,-,A,F,F,F,A,-,A,D
	-,-,D,C,F,F,F,F,F,F,F,D,A,-,A,F,-,F,A,-,A,D
	-,-,D,A,D,-,A,-,A,-,A,D,A,-,A,F,F,F,A,-,A,D
	-,-,-,D,D,-,A,-,A,-,A,-,D,A,-,A,A,A,-,A,D,-
	-,-,-,-,-,D,D,F,D,D,D,D,F,D,A,-,-,-,A,D,-,-
	-,-,-,-,-,-,D,A,D,-,-,D,-,-,D,A,A,A,D,-,-,-
	-,-,-,-,-,-,D,D,D,-,-,D,D,D,D,D,D,D,-,-,-,-
\end{filecontents*}

\PixlArtTikz[Codes=ABCDFJ,Symbols={1,2,3,4,6,10},Symb,Style=\tiny\sffamily,Unit=0.35]{cap.csv}
~~
\PixlArtTikz[Codes=ABCDFJ,Colors={red,brown,yellow,black,blue,gray},Correction,Unit=0.35]{cap.csv}
\end{PresentationCode}

\pagebreak

\subsection{Symbols within list(s)}

List of symbols can be given within list(s), picked randomly.

\begin{PresentationCode}{}
\begin{filecontents*}[overwrite]{testlist.csv}
A,B,C,A
A,B,B,C
B,A,C,B
C,A,B,C
\end{filecontents*}

\textbf{Notice : }

Multiples of 5 : red\\
Multiples of 3 : green\\
Multiples of 2 : blue

\PixlArtTikz[Codes=ABC,Symbols={5§25§35,3§9§21§27,2§4§8§14§16},Symb,Style=\large\sffamily,Unit=0.85]{testlist.csv}
\hspace{5mm}
\PixlArtTikz[Codes=ABC,Symbols={5§25§35,3§9§21§27,2§4§8§14§16},Symb,Style=\large\sffamily,Unit=0.85]{testlist.csv}
\hspace{5mm}
\PixlArtTikz[Codes=ABC,Colors={red,green,blue},Correction,Style=\large\sffamily,Unit=0.85]{testlist.csv}
\end{PresentationCode}

\pagebreak

\subsection{Starred macro}

The starred macro \texttt{\textbackslash PixlArtTikz*} is to be used within an already created environment. It can be useful for adding code after the PixelArt.

\smallskip

In this case:

\begin{itemize}
	\item the \textit{optional} argument between \texttt{<...>} is discarded;
	\item the key \Cle{Unit} is discarded too (units can be configured in the environment!)
\end{itemize}

\begin{PresentationCode}{}
\begin{center}
	\begin{tikzpicture}[scale=0.5]
		%grid to show positionning
		\draw[very thin,gray,xstep=1,ystep=1] (0,0) grid (17,-24) ;
		\foreach \x in {0,1,...,17} \draw[very thin,gray] (\x,-3pt)--(\x,3pt)%
		node[above,font=\scriptsize\sffamily] {\x} ;
		\foreach \y in {0,-1,...,-24} \draw[very thin,gray] (3pt,\y)--(-3pt,\y)%
		node[left,font=\scriptsize\sffamily] {\y} ;
		%le PixelArt
		\PixlArtTikz*[Codes=123469,Colors={red,brown,yellow,black,blue,white},Correction]{test1.csv}
		%added code
		\filldraw[blue] (14,-1) circle[radius=1] ;
		\filldraw[yellow] (14,-1) circle[radius=0.8] ;
		\draw[green,very thick,<-,>=latex] (15,-1) to[bend left=30] (18,-2)%
		node[right,font=\scriptsize\sffamily] {Code Ti\textit{k}Z} ;
	\end{tikzpicture}
\end{center}
\end{PresentationCode}

\pagebreak

\section{PixelArt environment}

\subsection{Usage}

The package \textsf{PixelArtTikz} provides an environment to create a PixelArt and add code afterwards.

\begin{itemize}
	\item The environment is created within Ti\textit{k}Z and additional code is passed on to the Ti\textit{k}Z environment!
	\item The additional code will be printed on top of the PixelArt!
\end{itemize}

\begin{PresentationCode}{listing only}
\begin{EnvPixlArtTikz}[keys]<options tikz>{filename.csv}
	%tikz code(s)
\end{EnvPixlArtTikz}
\end{PresentationCode}

The first argument, \textit{optional} and between \texttt{[...]}, proposes the \textsf{keys}:

\begin{itemize}
	\item the key \Cle{Codes} with the \textit{string} of \textit{simple} codes of the \textsf{csv} file;
	\item the key \Cle{Colors} with the \textit{list} of colors;
	\item the key \Cle{Symbols} with the \textit{optional list} of alt.\ symbols for the cells;
	\item the boolean \Cle{Correction} to color the PixelArt;\hfill{}default \textsf{false}
	\item the boolean \Cle{Symb} to print the symbols;\hfill{}default \textsf{false}
	\item the boolean \Cle{Border} to print borders of the cells;\hfill{}default \textsf{true}
	\item the key \Cle{Style} to specifythe style of the text. \hfill{}default \textsf{\textbackslash scriptsize}
\end{itemize}

The second argument, \textit{optional} and between \texttt{<...>}, is for Ti\textit{k}Z options to be passed on to the environment which creates the PixelArt.

\medskip

The third argument, \textit{mandatory}, is the filename of the \textsf{csv}.

\subsection{Example}

The symbols are at the nodes $(c\,;\,-l)$ where $l$ and $c$ are the row and column  of the data in the \textsf{csv} file.

\begin{PresentationCode}{}
\begin{center}
	\begin{EnvPixlArtTikz}%
			[Codes=123469,Colors={red,brown,yellow,black,blue,white},Correction,Unit=0.25]
			{test1.csv}
		\filldraw[blue] (14,-1) circle[radius=1] ;
		\filldraw[yellow] (14,-1) circle[radius=0.8] ;
		\draw[green,very thick,<-,>=latex] (15,-1) to[bend left=30] (18,-2)%
		node[right,font=\scriptsize\sffamily] {Ti\textit{k}Z code} ;
	\end{EnvPixlArtTikz}
\end{center}
\end{PresentationCode}

\newpage

\section{Macro for \textit{mini-}PixelArt}

\subsection{Idea}

The idea is tu propose a macro to insert, without \textsf{csv} file, a small PixelArt with small colors list.

\begin{PresentationCode}{listing only}
	\MiniPixlArt[keys]{list of colors}
\end{PresentationCode}

The first argument, \textit{optional} and between \texttt{[...]} proposes the \textsf{keys }:

\begin{itemize}
	\item the key \Cle{Unit} for dimension of the cells ;\hfill{}default \textsf{0.25em},
	\item the boolean \Cle{Border} to print a small border for the cells.\hfill{}default \textsf{false}
\end{itemize}

The second argument, \textit{mandatory} and between \texttt{\{...\}}, is the colors of the cells :

\begin{itemize}
	\item each color is \textit{coded} by a letter :
	\begin{multicols}{5}
		\begin{itemize}
			\item \texttt{R : \textcolor{red}{red}}
			\item \texttt{G : \textcolor{green}{green}}
			\item \texttt{C : \textcolor{cyan}{blue}}
			\item \texttt{Y : \textcolor{yellow}{yellow}}
			\item \texttt{B : \textcolor{black}{black}}
			\item \texttt{L : \textcolor{lightgray}{gray}}
			\item \texttt{. : \colorbox{lightgray}{\textcolor{white}{white}}}
			\item \texttt{M : \textcolor{brown}{maroon}}
			\item \texttt{O : \textcolor{orange}{orange}}
			\item \texttt{P : \textcolor{violet}{purple}}
		\end{itemize}
	\end{multicols}
	\item each linebreak is done by \texttt{,} ;
	\item the thickness of the borders are 10\,\% of the unit.
\end{itemize}

The last argument, \textit{optional} and between \texttt{<...>}, proposes options for the \textsf{tikz} environment.

\subsection{Examples}

\begin{PresentationCode}{}
\MiniPixlArt{%
	..RR..RR..,
	.RRRRRRRR.,
	RRRRRRRRRR,
	RRRRRRRRRR,
	RRRRRRRRRR,
	.RRRRRRRR.,
	..RRRRRR..,
	...RRRR...,
	....RR....
}
\end{PresentationCode}

\begin{PresentationCode}{}
Inline, we can give \MiniPixlArt[Unit=5mm,Border]{BCGOYG,YLP.BR}<baseline=(current bounding box.center)>
this miniPA.
\end{PresentationCode}

\newpage

\section{Macro for cutting PixelArt}

\subsection{Idea}

The idea is to offer commands to create \textit{collaborative} PixelArts, to form a \textit{large} image from several small ones (of the same size).

\smallskip

{\footnotesize\faBomb} For practical reasons (related to the use of \texttt{filecontents}) the file \textsf{csv} must end with an empty line (it is created automatically via \texttt{filecontents}, and the code uses this specificity).

\smallskip

The available \Cle{keys} are exactly the same as those in classic displays.

\subsection{Global usage}

The argument \texttt{cutting} can be given in the form :
%
\begin{itemize}
	\item \Cle{<number of block length>x<number of block col>} to specify the size of the blocks;
	\item \Cle{<number of V blocks>+<number of H blocks>} to specify the number of blocks.
\end{itemize}

\begin{PresentationCode}{listing only}
\CutPixlArtTikz(*)[keys]<tikz options>{file.csv}{cutting}
%(*) := change layout formatting (A1 or 1.1)
%1   := keys
%2   := tikz options
%3   := csv file
%4   := block size (LxC) or block number (L+C)

\HelpGridPixelArtTikz(*)[scale]{file.csv}{cutting}
%(*) := change layout formatting (A1 or 1.1)
%1   := scale of notice
%2   := csv file
%4   := block size (LxC) or block number (L+C)
\end{PresentationCode}

\begin{filecontents*}[overwrite]{PAkoala.csv}
A,A,A,A,A,A,A,H,H,H,A,A,A,A,A,H,H,H,A,A,A,A,A,A,A,A,A,H,H,H,H,A,A,H,A,A,A,A,A,A
A,A,A,A,A,A,A,A,H,A,A,A,A,A,A,H,H,A,A,A,A,A,A,A,A,A,A,H,A,A,A,A,A,H,A,A,A,A,A,A
A,A,A,H,A,A,A,A,A,A,A,A,A,A,A,H,H,A,A,A,A,A,A,A,A,H,H,H,A,A,A,A,A,A,A,A,A,A,A,A
A,A,A,A,A,A,A,B,E,E,E,E,A,A,A,A,A,A,A,A,A,A,A,H,H,H,A,B,B,E,E,E,E,B,A,A,A,A,A,H
A,A,H,A,A,A,A,B,E,E,E,E,E,A,A,A,A,A,A,A,A,A,A,H,H,H,B,E,E,E,E,E,E,B,B,A,A,A,A,H
A,A,H,A,A,B,B,E,C,C,C,C,C,E,B,B,B,E,E,E,E,E,E,B,B,B,E,C,C,C,C,C,C,E,B,B,H,H,A,A
A,A,H,A,B,E,D,D,D,D,B,D,C,C,E,E,E,E,C,C,C,C,E,E,E,E,E,C,D,B,B,D,D,D,E,B,H,A,A,A
A,A,A,A,B,E,D,D,F,F,D,B,D,C,C,E,C,C,C,C,C,C,C,C,E,C,C,D,B,D,D,F,F,D,E,B,A,A,A,H
A,A,A,B,C,D,D,F,F,F,D,D,B,B,C,C,C,C,C,C,C,C,C,C,C,C,C,B,D,D,D,F,F,D,D,C,B,A,H,H
A,A,A,B,C,D,F,F,F,F,F,D,D,D,C,C,C,C,C,C,C,C,C,C,C,C,C,D,D,F,F,F,F,F,D,C,B,A,H,H
A,A,A,B,D,D,F,F,F,F,F,D,C,C,C,C,C,C,C,C,C,C,C,C,C,C,C,C,D,F,F,F,F,F,D,D,B,A,A,H
A,A,A,B,D,D,G,F,F,F,F,D,C,C,B,B,B,C,C,B,B,C,C,B,B,B,C,C,D,F,F,F,F,G,D,D,B,A,A,A
A,A,A,A,B,G,G,F,F,F,F,D,C,C,B,F,B,C,B,D,D,B,C,B,F,B,C,C,D,G,F,F,F,G,G,B,H,A,A,A
A,A,A,A,A,B,G,G,G,F,G,D,C,C,C,B,B,C,B,D,D,B,C,B,B,C,C,C,D,G,G,F,G,G,B,H,H,H,A,A
A,A,A,A,A,A,B,G,G,G,G,B,C,C,C,C,C,C,B,B,B,B,C,C,C,C,C,C,D,B,G,G,G,G,B,H,H,H,A,A
A,A,A,A,A,A,A,B,B,B,B,B,C,C,C,C,C,C,B,B,B,B,C,C,C,C,C,C,D,B,B,B,B,B,A,A,H,H,A,A
A,A,A,A,A,A,A,A,A,A,A,D,D,C,C,C,C,F,B,B,B,B,F,C,C,C,C,D,D,A,A,A,A,A,A,A,A,A,A,A
A,A,A,A,A,A,A,A,A,A,A,B,D,D,C,C,C,F,F,B,B,F,F,C,C,C,D,D,B,A,A,A,A,A,A,A,A,A,A,A
A,A,A,A,A,H,H,H,A,A,A,A,B,B,D,C,C,F,F,F,F,F,F,C,C,D,B,B,A,A,A,A,A,A,H,H,H,A,A,A
A,A,A,A,H,H,H,H,A,A,A,A,B,B,D,D,D,G,G,G,G,G,G,D,D,D,D,B,A,A,A,A,A,A,A,H,H,H,A,A
A,A,A,A,H,H,H,A,A,A,A,B,D,D,D,D,D,G,G,G,G,G,G,D,D,D,D,D,B,A,A,A,A,A,A,A,H,H,A,A
A,A,A,H,A,A,A,A,A,A,B,C,D,D,D,G,G,G,D,D,D,D,G,G,G,D,D,D,C,B,A,H,H,A,A,A,A,H,A,A
A,A,A,H,H,A,A,A,A,A,D,C,C,C,D,G,G,F,G,D,D,G,F,G,G,D,D,C,C,C,B,A,H,H,A,A,A,A,A,A
A,A,A,A,A,H,H,H,H,B,D,D,C,C,D,G,G,F,F,F,F,F,F,G,G,D,D,C,C,D,B,A,A,H,H,A,A,A,A,A
A,A,A,A,A,A,A,A,A,B,D,D,C,C,C,C,C,B,F,F,F,F,F,B,C,C,C,C,D,D,B,A,A,H,H,H,H,A,A,A
A,A,A,H,H,A,A,A,A,B,D,D,C,C,C,C,C,C,B,F,F,F,B,C,C,C,C,C,D,D,B,A,A,A,A,H,H,A,A,A
A,A,A,A,A,A,A,A,A,A,B,B,D,D,C,C,C,C,C,B,F,B,C,C,C,C,D,D,B,B,A,A,A,H,H,A,A,A,A,A
A,A,A,A,A,A,A,A,A,A,B,B,B,D,D,C,C,C,C,B,F,B,C,C,C,C,D,B,B,B,A,A,A,H,H,A,A,A,A,A
A,A,A,A,A,A,A,A,A,B,C,C,C,B,B,D,D,D,D,B,F,B,D,D,D,B,B,D,C,C,B,A,A,A,H,A,A,A,A,A
A,A,A,A,A,A,A,A,B,C,C,C,C,C,D,B,B,B,F,F,F,F,B,B,B,D,D,C,C,C,C,B,A,A,A,A,A,A,H,H
A,A,A,A,A,A,A,A,B,C,C,C,C,C,D,D,D,G,G,F,F,G,G,D,D,D,D,C,C,C,C,B,A,A,A,A,A,A,H,H
A,A,A,A,A,A,A,I,B,D,C,C,C,C,C,C,C,G,G,F,F,G,G,C,C,C,C,C,C,C,D,B,I,A,A,A,A,A,A,A
A,A,A,A,A,A,A,I,B,D,D,D,C,C,C,B,B,B,G,G,G,G,B,B,B,C,C,C,D,D,D,B,I,A,A,A,A,A,A,A
A,A,A,A,A,H,A,I,I,B,D,D,D,D,D,C,C,C,B,D,D,B,C,C,C,C,D,D,D,B,B,I,I,A,A,A,A,A,A,A
A,A,A,A,A,H,H,H,I,B,B,D,D,D,D,D,C,C,B,D,D,B,C,C,C,D,D,D,D,B,I,I,A,A,A,A,A,A,A,A
A,A,H,H,H,A,A,A,I,I,I,B,B,B,D,B,B,B,B,B,B,B,B,B,B,D,B,B,B,I,I,I,A,A,A,A,A,A,H,A
A,A,H,H,H,A,A,A,A,I,I,I,B,B,B,B,B,B,B,B,B,B,B,B,B,B,B,B,I,I,I,A,A,A,A,A,A,A,H,A
H,H,A,H,H,A,A,A,A,A,A,I,I,I,I,I,I,I,I,I,I,I,I,I,I,I,I,I,I,I,A,A,A,A,A,A,A,A,A,A
A,A,A,A,A,A,A,A,A,A,A,A,A,A,A,I,I,I,I,I,I,I,I,I,A,A,A,A,A,A,A,A,A,A,A,H,A,A,A,A
A,A,A,A,A,A,A,A,A,A,A,A,H,H,A,A,A,A,A,A,A,A,A,A,A,A,A,A,A,A,A,A,A,A,H,H,A,A,A,A
\end{filecontents*}

\subsection{Example}

\def\listcoulkoala%
{VertForet,Noir,GrisClair,GrisFonce,Beige,Blanc,BleuClair,VertClair,Marron}

\begin{PresentationCode}{listing only}
\def\listcoulkoala%
	{VertForet,Noir,GrisClair,GrisFonce,Beige,Blanc,BleuClair,VertClair,Marron}
\end{PresentationCode}

\begin{PresentationCode}{}
%block with 8x8 size
\HelpGridPixelArtTikz{PAkoala.csv}{8x8}
~~
\PixlArtTikz[Correction,Unit=0.125,Codes=ABCDEFGHI,Colors={\listcoulkoala},Grid=8x8]{PAkoala.csv}
\end{PresentationCode}

\begin{PresentationCode}{}
%16 blocks (4L 4C)
\HelpGridPixelArtTikz*[2]{PAkoala.csv}{4+4}
~~
\PixlArtTikz[Correction,Unit=0.2,Codes=ABCDEFGHI,Colors={\listcoulkoala},Grid=4+4]{PAkoala.csv}
\end{PresentationCode}

\newpage

\begin{PresentationCode}{}
%block with 8x8 size
\CutPixlArtTikz[Unit=0.3,Codes=ABCDEFGHI]{PAkoala.csv}{8x8}
\end{PresentationCode}

\pagebreak

\begin{PresentationCode}{}
%16 blocks (4L 4C)
\CutPixlArtTikz*[Unit=0.3,Codes=ABCDEFGHI,Correction,Colors={\listcoulkoala}]{PAkoala.csv}{4+4}
\end{PresentationCode}

\pagebreak

\section{Anamorphic cylinder PixelArt}

\subsection{Idea}

The idea is to provide something to create PixelArt with the aim of using a cylindrical anamorphosis.

On \url{https://www.youtube.com/watch?v=PT8KUozBg3I}, there is a video \textit{demonstration}, proposed by Jean-Yves Labouche.

The overall operation is similar to that of the \textit{main} command, however there are some adjustments:

\begin{itemize}
	\item the possibility of giving the file \textsf{csv} in \textit{normal} or \textit{inverted} mode;
	\item the dimensions (width \& middle) must be specified to produce the PixelArt;
	\item the command is standalone (for the moment) so no addition(s) later.
\end{itemize}

\begin{PresentationCode}{listing only}
\PixlArtTikzCylindric[keys]{file.csv}
\end{PresentationCode}

The files illustrating this paragraph are given below.

\begin{PresentationCode}{}
%version with inverted data
\begin{filecontents*}[overwrite]{PAyoda.csv}
E,E,E,A,A,E,A,A,E,D
E,E,E,F,F,F,F,F,E,D
E,E,E,F,F,C,F,F,E,D
E,A,F,F,C,C,C,F,F,A
E,A,F,F,C,C,C,F,F,A
E,E,E,A,A,A,A,A,E,E
E,E,E,A,A,A,A,A,E,E
E,E,A,E,B,A,B,E,A,E
E,A,A,A,A,A,A,A,A,A
E,E,E,A,A,A,A,A,E,E
\end{filecontents*}

\PixlArtTikz[%
	Codes=ABCDEF,
	Colors={green,black,brown,white,brown!75!black,lightgray},
	Correction,Unit=0.35]%
	{PAyoda.csv}
\end{PresentationCode}

\begin{PresentationCode}{}
%version with normal data
\begin{filecontents*}{PAsorciere.csv}
V,V,V,V,N,N,N,V,V,V
V,V,V,N,N,N,N,N,V,V
N,N,N,N,N,N,N,N,N,N
V,V,V,G,G,G,O,O,V,V
V,V,V,J,J,G,O,O,O,V
V,V,G,N,J,G,O,O,O,V
V,G,G,G,G,G,O,O,O,O
V,V,N,N,N,G,O,O,O,O
V,V,G,G,G,G,O,O,O,O
V,V,V,V,N,N,O,O,O,O
\end{filecontents*}

\PixlArtTikz[%
	Codes=VNGOJ,
	Colors={purple!50,black,green,orange,yellow},
	Correction,Unit=0.35]%
	{PAsorciere.csv}
\end{PresentationCode}

\subsection{Keys and options}

The first argument, \textit{optional} and between \texttt{[...]} offers \textsf{Keys} necessary for the command to function properly:

\begin{itemize}
	\item the key \Cle{Width} which defines the width (radius in cm) of the rendering;\hfill{}default \textsf{6}
	\item the key \Cle{Center} which defines the width (radius in cm) of the \textit{middle};\hfill{}default \textsf{1.25}
	\item the key \Cle{Codes} contains the \textit{string} of the \textit{simple} codes of the file \textsf{csv};
	\item the key \Cle{Colors} which contains the \textit{list} of associated colors;
	\item the key \Cle{Symbols} which contains the \textit{possible list} of alternative characters to display in the boxes;
	\item the key \Cle{Style} which allows you to specify the style of characters. \hfill{}default \textsf{normalsize}
	\item the Boolean key \Cle{Correction} which allows you to color the PixelArt;\hfill{}default \textsf{false}
	\item the Boolean key \Cle{Symb} which allows you to display the \textit{alternative} characters;\hfill{}default \textsf{false}
	\item the Boolean key \Cle{Solution} which displays the solution (with \textit{mirror} effect);\hfill{}default \textsf{false}
	\item the Boolean key \Cle{Swap} which allows you to specify the data type (\Cle{true} := normal; \Cle{false} := inverted).
	
	\hfill{}default \textsf{false}
\end{itemize}

The second argument, \textit{mandatory}, is the name of the \textsf{csv} file to use.

\pagebreak

\subsection{Example with inverted data (Yoda)}

In this paragraph, we use the data \texttt{PAyoda}, which corresponds to the \textit{inverted} layout, so the key \Cle{Swap} is not necessary.

\begin{PresentationCode}{}
%classic version
\PixlArtTikzCylindric[Codes=ABCDEF,Style=\small\ttfamily]{PAyoda.csv}

%version with 'symbols'
\PixlArtTikzCylindric[Codes=ABCDEF,Symbols={1,2,3,4,5,6},Symb,Style=\small\ttfamily]{PAyoda.csv}
\end{PresentationCode}

\begin{PresentationCode}{}
%Fix and solution
\begin{tabular}{c}
	\PixlArtTikzCylindric[%
		Codes=ABCDEF,
		Colors={green,black,brown,white,brown!75!black,lightgray},
		Solution]%
		{PAyoda.csv}
	\\
	\PixlArtTikzCylindric[%
		Codes=ABCDEF,
		Colors={green,black,brown,white,brown!75!black,lightgray},
		Correction]%
		{PAyoda.csv}
\end{tabular}
\end{PresentationCode}

\pagebreak

\subsection{Example with classic data (Witch)}

In this paragraph, we use the data \texttt{PAsorciere}, which corresponds to the \textit{normal} layout, so the key \Cle{Swap} is necessary.

\begin{PresentationCode}{}
	%classic version
	\PixlArtTikzCylindric[%
	Width=4,Center=1,Codes=VNGOJ,
	Colors={purple!50,black,green,orange,yellow},
	Swap,Style=\ttfamily\scriptsize]%
	{PAsorciere.csv}
\end{PresentationCode}

\begin{PresentationCode}{}
%Correction and solution
\begin{tabular}{c}
	\PixlArtTikzCylindric[%
		Width=4,Center=0.75,Codes=VNGOJ,
		Colors={purple!50,black,green,orange,yellow},
		Swap,Solution]%
		{PAsorciere.csv}
	\\
	\PixlArtTikzCylindric[%
		Width=4,Center=0.75,Codes=VNGOJ,
		Colors={purple!50,black,green,orange,yellow},
		Swap,Correction]%
		{PAsorciere.csv}
\end{tabular}
\end{PresentationCode}

\newpage

\part{History}

\verb|v0.1.5|~:~~~~Symbols can be given with list(s) + Enhancements of internal code

\verb|v0.1.4|~:~~~~Anamorphic cylinder PixelArts

\verb|v0.1.3|~:~~~~Cut PixelArts within several PixelArts

\verb|v0.1.2|~:~~~~\textit{mini-}PixelArts

\verb|v0.1.1|~:~~~~Bugfix with color

\verb|v0.1.0|~:~~~~Initial version

\end{document}